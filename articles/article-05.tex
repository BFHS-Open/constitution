\article{Impeachment, Resignation}

\asection{Impeachment}
\begin{enumerate}
	\item To initiate an Impeachment, an Individual must:
		\begin{enumerate}
			\item Have Contributing Membership.

			\item Be against an Individual of Executive Membership.

			\item Approval by sponsor.
		\end{enumerate}

	\item After initiated:
		\begin{enumerate}
			\item There will be public vote at the next in-person meeting, in which
				present members will vote on impeaching said Executive Member (\refsection{Voting Method}).
		\end{enumerate}

	\item Once Impeachment passes:
		\begin{enumerate}
			\item The Executive Member will be stripped of their position on the Executive
				Committee and will be returned to any previous Membership position.

			\item A new Individual will be elected through the method described in the
				description of the individual Executive Board position (\refarticle{Executive Board}).
		\end{enumerate}

	\item On the occasion that there are multiple successful Impeachments within one
		Academic Year:
		\begin{enumerate}
			\item A new Snap Election (\refsection{Snap Election}) will be held for
				all Executive Board positions.

			\item At the ending of the Snap Election, the counter will be reset to 0.
		\end{enumerate}
\end{enumerate}

\asection{Direct Resignation}
\begin{enumerate}
	\item To initiate a Direct Resignation, an Individual must:
		\begin{enumerate}
			\item Consensually and prematurely end their position within the Organization.

			\item Optionally, they may give a reason to the wider Organization.
		\end{enumerate}

	\item After the Direct Resignation has been initiated:
		\begin{enumerate}
			\item A notice for Direct Resignation is required of some Membership positions,
				and are described as such in their respective sections (\refarticle{Involvement}).
		\end{enumerate}

	\item Once the notice has expired:
		\begin{enumerate}
			\item The Individual will be stripped of the position they are resigning for.

			\item If the Individual wishes to retain their Membership in the Organization,
				they will be allowed to re-register for Membership.
		\end{enumerate}
\end{enumerate}

\asection{Failure to Meet Requirements}
\begin{enumerate}
	\item Non-enrollment at Benjamin Franklin High School:
		\begin{enumerate}
			\item If the individual is not actively enrolled at Benjamin Franklin High
				School, they will be barred from participating in Organization events
				unless invited.

			\item They are still allowed to advertise any past or present affiliation with
				the Organization.
		\end{enumerate}

	\item Academic probation:
		\begin{enumerate}
			\item If the individual is under academic probation, they will be barred
				from events and will be demoted from any held positions.

			\item Executive Committee members will be demoted and a replacement will be
				found as described by their job title (\refarticle{Executive Board}).

			\item Committee Members will be demoted, but they can request their
				position again after their probation has ended.
		\end{enumerate}
\end{enumerate}

\asection{Expulsion}
\begin{enumerate}
	\item To initiate an Expulsion, and Individual must:
		\begin{enumerate}
			\item Have evidence of another Individual committing an egregious violation
				of the Code of Conduct, School Policy, or a Moral Failing.

			\item Report said evidence to the Organization's Sponsor.
		\end{enumerate}

	\item After Expulsion has been initiated:
		\begin{enumerate}
			\item The sponsor will investigate the evidence and either decide to continue
				or end the Expulsion proceeding.
		\end{enumerate}

	\item If an Individual has been expelled:
		\begin{enumerate}
			\item They will no longer be allowed to associate with the Organization.

			\item They will be removed and excluded from all Organization events and activities.

			\item They will be reported to Benjamin Franklin High School’s Administration
				if there are grounds for it.

			\item They will not be allowed back into the Organization under any circumstances.
		\end{enumerate}
\end{enumerate}