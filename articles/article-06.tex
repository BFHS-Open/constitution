\article{Voting}

\asection{Voting Eligibility}
\begin{enumerate}
	\item To vote in General Elections, one must have Contributing Membership (\refsection{Active Membership}).

	\item Committee Voting Eligibility requires:
		\begin{enumerate}
			\item Executive Membership and accompanying Executive Board position (\refsection{Executive Committee}).
		\end{enumerate}

	\item General Members may be privy to the elections, if desired, but will have
		no vote.
\end{enumerate}

\asection{Voting Method}
\begin{enumerate}
	\item The voting method is closest to ``Single-Member Plurality Voting''.

	\item An issue is raised and voted upon:
		\begin{enumerate}
			\item An Individual up for Election (Candidate).

			\item Other issues (\refsection{Executive Committee}).
		\end{enumerate}

	\item If the issue raised is an Election, all participating Individuals will
		be granted one of these choices:
		\begin{enumerate}
			\item Vote for one of the Candidates.

			\item Abstain.
		\end{enumerate}

	\item If the issue is raised is not an Election, all participating Individuals
		will be granted one of these choices:
		\begin{enumerate}
			\item Aye (Agree).

			\item Abstain (No Vote).

			\item Nay (Disagree).
		\end{enumerate}

	\item Once all participating Individuals have completed their voting:
		\begin{enumerate}
			\item All votes will be tallied up and the option voted upon the most will
				be enforced.

			\item All ``Abstain'' votes will be ignored.
		\end{enumerate}
\end{enumerate}

\asection{General Election}
\begin{enumerate}
	\item To a initiate a General Election:
		\begin{enumerate}
			\item Within first 6 weeks of the Academic Year.

			\item At least one Candidate for each position voted upon.
		\end{enumerate}

	\item To register as a Candidate:
		\begin{enumerate}
			\item State your Candidacy and gain approval from the Organization's Sponsor.

			\item Current Active Membership within the Organization.

			\item A Candidate can not run for multiple positions at the same time.
		\end{enumerate}

	\item Once a General Election has been initiated:
		\begin{enumerate}
			\item A General Election will be scheduled one week later.

			\item All Candidates will be voted upon for their respective positions (\refsection{Voting Method}).

			\item Once all votes have been cast, the Candidate with the most votes for
				their respective position will win.
		\end{enumerate}

	\item Once a General Election has ended:
		\begin{enumerate}
			\item The previous Executive Board will hand-off control to the incoming Executive
				Board.

			\item The Sponsor will verify all goes smoothly.
		\end{enumerate}
\end{enumerate}

\asection{Snap Election}
\begin{enumerate}
	\item A Snap Election is initiated through multiple impeachments within one Academic
		Year (\refsection{Impeachment}).

	\item Once the Snap Election has been initiated:
		\begin{enumerate}
			\item Potential Candidates must gain approval from the Organization's Sponsor.

			\item The Snap Election will be scheduled within the next week.
		\end{enumerate}

	\item Just like a General Election:
		\begin{enumerate}
			\item All positions with Candidates will be voted upon.

			\item At the end, all votes will be tallied up and the Candidate with the
				most for each position will be elected.
		\end{enumerate}

	\item After the Snap Election ended:
		\begin{enumerate}
			\item The Sponsor will have to enforce the new Election results and force a
				transfer of power.
		\end{enumerate}
\end{enumerate}

\asection{Committee Votes}
\begin{enumerate}
	\item Committee Votes are votes that are brought by and managed by the Executive
		Committee.

	\item Before voting commences, the Executive Committee must:
		\begin{enumerate}
			\item Publicize what the Executive Committee Board is voting upon.
		\end{enumerate}

	\item The Voting Method is same as previously defined (\refsection{Voting Method}).

	\item Once voting had ended:
		\begin{enumerate}
			\item Results will be publicized and the changes votes enacted.
		\end{enumerate}
\end{enumerate}